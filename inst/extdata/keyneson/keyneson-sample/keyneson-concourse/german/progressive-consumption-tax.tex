Leute sollten Steuern zahlen als progressiven Anteil auf die von ihnen konsumierten Güter und Dienstleistungen, definiert als die Differenz zwischen ihren Einkommen und Netto-Ersparnissen.
Zum Beispiel: Eine Person die € 50.000 verdient, € 20.000 gespart, und € 10.000 von Ersparnissen aufgebraucht hat, muss € 40.000 konsumiert haben, und sollte darauf besteuert werden.